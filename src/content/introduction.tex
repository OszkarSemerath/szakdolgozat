%----------------------------------------------------------------------------
\chapter{\bevezetes}
%----------------------------------------------------------------------------

\section{Témamegjelölés}
Modell-vezérelt tervezésnek nagy szerepe van az informatikában egy szoftver vagy rendszer létrehozásánál. Modellek segítségével sokkal jobban, strukturáltabban és megbízhatóbban lehet tervezni. Ehhez a rengeteg eszköz áll rendelkezésünkre, viszont ezek nem mindenre nyújtanak megoldást.
\section{Problémafelvetés}
Mai modellező eszközökkel nem lehetséges, hogy részleges, vagy hibás modelleket kezeljünk, elmentsünk. A modell készítésének közben lehetnek döntések, amik bizonytalanok, és nem is feltétlen fontos velük foglalkozni abban a tervezési szakaszban. Ezeket az elemeket célszerű lenne valamilyen módon megjelölni. Előfordulhat hogy a modellben egy elem megléte, vagy a multiplicitása kétséges. Esetleg, olyan, hogy nem lehet eldönteni egy elemről hogy az melyik másikkal áll kapcsolatban.
\section{Célkitűzés}	
Kutatásom célja, olyan domain független modell elkészítése, aminek segítségével lehet részleges, vagy hiányos modelleket készíteni. Például, előfordulhat olyan, hogy nem tudjuk eldönteni egy attribútumról, hogy melyik elemhez tartozik. Ilyet általában a modellező eszközök nem támogatnak. Célszerű lenne egy olyan általános módszert kitalálni, amivel lehetséges az ilyen és ehhez hasonló esetek megjelölése, kezelése.
\section{Kontribúció}
Kutatásom során megismerkedtem a részleges modellezés leglényegesebb aspektusaival. Létrehoztam egy olyan metamodellt, ami képes részleges modellek készítésére. Ezután elkészítettem egy olyan vizuális szerkesztőfelületet, aminek a segítségével részleges modell példányokat lehet készíteni.  

\section{Hozzáadott érték}	

\section{Dolgozat felépítése}

